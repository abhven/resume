%-------------------------------------------------------------------------------
%	SECTION TITLE
%-------------------------------------------------------------------------------
\cvsection{Education}


%-------------------------------------------------------------------------------
%	CONTENT
%-------------------------------------------------------------------------------
\begin{cventries2}
%---------------------------------------------------------
\cventry
{Master's in Robotics} % Job title
{University of Michigan} % Organization
{Ann Arbor, Michigan} % Location
{} % Date(s)
{}
\vspace{-0.3cm}
\cventry
{Graduate Student Research Associate} % Job title
{} % Organization
{} % Location
{} % Date(s)
{
	\begin{cvitems} % Description(s) of tasks/responsibilities
%		\item {Kinematically-informed Interactive perception \href{https://arxiv.org/pdf/1901.05580}{[Link]}, enables robots to classify objects by building 3D models through observations from a RGBD camera while manipulating the object. Method was validated on Toyota’s HSR Robot}
%		\vspace{0.1cm}
%		\item SPARE {\href{https://arxiv.org/pdf/1803.11147}{[Link]}, extendable articulated object RGBD dataset. Method could randomize configurations on Gazebo and provide simulated observations from multiple viewpoints. Implemented DNN (Conv3D/Conv+LSTM, FC) to count links and estimate link lengths on Tensorflow}
\item{
	Enabled Personal Robots (Fetch/ HSR) to interact using DNN. Papers:
	\begin{itemize}
	\item {Kinematically-informed Interactive perception \href{https://arxiv.org/pdf/1901.05580}{[Link]}}
	\vspace{0.1cm}
	\item SPARE - an extendable articulated object RGBD dataset {\href{https://arxiv.org/pdf/1803.11147}{[Link]}}
	\end{itemize}
	}
	\end{cvitems}
}

%---------------------------------------------------------
\cventry
{Graduate Student Instructor} % Job title
{} % Organization
{} % Location
{} % Date(s)
{
	\begin{cvitems} % Description(s) of tasks/responsibilities
		\item {\emph{Mathematics for Robotics} - linear algebra including, vector spaces, orthogonal basis, SVD, QR Factorization, BLUE, MVE and Kalman Filters}
		\item {\emph{Advanced Topics in Computer Vision} - pictorial structures, graphical models, CNN, GAN, RNN, LSTM, auto-encoders. }
	\end{cvitems}
}
%---------------------------------------------------------
\vspace{0.2cm}
\cventry
{B.Tech in Mechanical Engineering and M.Tech in Product Design} % Degree
{Indian Institute of Technology Madras} % Institution
{Chennai, India} % Location
{} % Date(s)
{}
\vspace{-0.5cm}
%---------------------------------------------------------
\end{cventries2}




